\section{Regression on California Housing}

After the data was loaded and split in training and test sets, feature normalization was performed on the training set. Using the statistics learned from the training set, the test set was normalized in the same way.

The training hyperparameters used for both libraries are shown in the table below.
\begin{table}[h]
\centering
\begin{tabular}{|c|c|c|}
    \hline
    Hyperparameter & \mfnet & \pytorch \\
    \hline
    Number of Epochs & 500 & 500 \\
    Learning Rate $\eta$ & 0.001 & 0.001 \\
    Batch Size & 1024 & 1024 \\
    \hline
\end{tabular}
\end{table}

For each library, three models were trained and compared:
\begin{enumerate}
    \item a baseline mean predictor that always predicts the mean value of the training set;
    \item a linear regression model;
    \item a neural network with one hidden layer of 512 neurons and ReLU activation functions. For \mfnet only, maximum gradient norm was set to 5 in order to prevent overflow errors.
\end{enumerate}

The learning curves for both libraries are shown in \cref{fig:regr_mfnet,fig:regr_pytorch}.

\begin{figure}
    \centering
    \includegraphics[width=0.75\linewidth]{Images/mfnet_regr_mserror_500_0.001_1024.png}
    \caption{Learning curves for regression task on California Housing dataset using \mfnet.}
    \label{fig:regr_mfnet}
\end{figure}

\begin{figure}
    \centering
    \includegraphics[width=0.75\linewidth]{Images/pytorch_regr_mserror_500_0.001_1024.png}
    \caption{Learning curves for regression task on California Housing dataset using \pytorch.}
    \label{fig:regr_pytorch}
\end{figure}
