\section{Classification on MNIST}

After loading the data (already split in training and test sets), the pixel values were normalized to the range [0, 1]. \mfnet also required the data to be transformed in such a way that the library would be able to understand. This meant converting input data to tensors and one-hot encoding the target labels. For both libraries the $28\times\SI{28}{\pixel}$ images were flattened into 784-dimensional vectors.

The training hyperparameters used for both libraries are shown in the \cref{tab:class_hyperparams}.
\begin{table}[ht]
\centering
\begin{tabular}{|c|c|c|}
    \hline
    Hyperparameter & \mfnet & \pytorch \\
    \hline
    Number of Epochs & 100 & 100 \\
    Learning Rate $\eta$ & 0.001 & 0.01 \\
    Batch Size & 1024 & 128 \\
    \hline
\end{tabular}
\caption{Training hyperparameters for classification task.}
\label{tab:class_hyperparams}
\end{table}

For each library, two models were trained and compared:
\begin{enumerate}
    \item a linear classification model;
    \item a neural network with one hidden layer of 512 neurons and ReLU activation functions. For \mfnet only, maximum gradient norm was set to 5 in order to prevent overflow errors.
\end{enumerate}

The learning curves and test accuracies for both libraries are shown in \cref{fig:class_mfnet,fig:class_pytorch}.

\begin{figure}[ht]
    \centering
    \includegraphics[width=0.75\linewidth]{Images/mfnet_class_loss_acc_100_0.001_1024.png}
    \caption{Learning curves for classification task on MNIST dataset using \mfnet.}
    \label{fig:class_mfnet}
\end{figure}

\begin{figure}[ht]
    \centering
    \includegraphics[width=0.75\linewidth]{Images/pytorch_class_loss_acc_100_0.01_128.png}
    \caption{Learning curves for classification task on MNIST dataset using \pytorch.}
    \label{fig:class_pytorch}
\end{figure}

The accuracy achieved by \pytorch is significantly lower than the one achieved by \mfnet, so the confusion matrices for both libraries are computed and shown in \cref{fig:conf_mfnet,fig:conf_pytorch}. The figures clearly show that, while \mfnet is able to correctly classify almost all the digits, \pytorch struggles with most of them (particularly the digits 0 and 8). This is likely due to insufficient exploration of the hyperparameter space, which wasn't carried out since it was outside the scope of this project.

\begin{figure}[ht]
    \centering
    \includegraphics[width=0.85\linewidth]{Images/mfnet_class_conf_matrix_100_0.001_1024.png}
    \caption{Confusion matrix for classification task on MNIST dataset using \mfnet.}
    \label{fig:conf_mfnet}
\end{figure}

\begin{figure}[ht]
    \centering
    \includegraphics[width=0.85\linewidth]{Images/pytorch_class_conf_matrix_100_0.01_128.png}
    \caption{Confusion matrix for classification task on MNIST dataset using \pytorch.}
    \label{fig:conf_pytorch}
\end{figure}
