\section{\acl*{NN}, Optimizer and Dataloader}
In this section we describe the implementation of more high-level objects that can be built using the building blocks described in the previous sections.

\subsection{\acl*{NN}}
A \acl{NN} is made up of a sequence of layers, each one transforming the input tensor into an output tensor, and each one feeding into the next.

As far as coding is concerned, it is a very simple object, and only defines three methods: forward, backward and a method that yields an iterator on all the layers' parameters. This last method is used by the optimizer when performing the update step.

The forward method simply calls the forward method of each layer in sequence, passing the output of one layer as input to the next one.

The backward method does the opposite: it calls the backward method of each layer in reverse order, passing the gradient of the loss with respect to the input of each layer as input to the previous layer.

\subsection{Optimizer}
The optimizer is responsible for updating the parameters of the \acl{NN} during training.

The most basic optimizer is \ac{GD}, which updates the parameters according to the rule:
\begin{equation}
    W^{[l]} \gets W^{[l]} - \eta \pdv{\J}{W^{[l]}}
\end{equation}
where $\eta$ is the learning rate.

This algorithm in this form suffers from many problems, such as having a big computational overhead and not being able to escape local minima. A more robust version is \ac{SGD}, which updates the parameters using a mini-batch (see \cref{sec:dataloader}) instead of the whole dataset. This introduces some stochastic noise in the updates, which can help the optimizer to better explore the space. It also reduces the size of the gradient matrix that needs to be computed, greatly reducing the computational burden.

\paragraph{Implementation details} When training a full \ac{NN} using ReLUs as activation functions, there was often an overflow problem. To help mitigate this, the \mono{Optimizer} class implements gradient clipping, which prevents the gradients from becoming too large by clipping the norm of the gradient tensor to a maximum value provided by the user.

\subsection{Dataloader} \label{sec:dataloader}
The dataloader is responsible for loading the data in mini-batches and shuffling it at the beginning of each epoch. This is important for \ac{SGD} to work properly, as it helps to reduce the correlation between consecutive mini-batches and improves the convergence of the optimizer.

\paragraph{Implementation details} In addition to creating the mini-batches, the\\\mono{DataLoader} class is also responsible for transforming the data, which is usually stored in a \shape{m}{n_\text{features}} matrix (the design matrix), into a format compatible with the \mfnet library: first, the design matrix is transposed, so that each column represents a data sample and each row represents a feature; then, the bias feature is prepended to the input data as a row of 1s. The same transformations are applied to the target data.

\paragraph{Example}
For a complete step-by-step example of how the learning process works, please refer to \cref{sec:example}.
